%!TEX program = xelatex
% 编译顺序: xelatex -> bibtex -> xelatex -> xelatex
% 国家自然科学基金NSFC面上项目申请书正文模板(2023年版)version1.0
% 声明:
% 注意!!!非国家自然科学基金委官方模版!!!由个人根据官方MsWord模版制作。本模版的作者尽力使本模版和官方模版生成的PDF文件视觉效果大致一样,然而,并不保证本模版有用,也不对使用本模版造成的任何直接或间接后果负责。 不得将本模版用于商用或获取经济利益。本模版可以自由修改以满足用户自己的需要。但是如果要传播本模版,则只能传播未经修改的版本。使用本模版意味着同意上述声明。
% 强烈建议自己对照官方MsWord模板确认格式和文字是否一致,尤其是蓝字部分。
% 如有问题,可以发邮件到ryanzz@foxmail.com



\documentclass[12pt,UTF8,AutoFakeBold=3,a4paper]{ctexart} %默认小四号字。允许楷体粗体。
\usepackage{nsfc}

\usepackage{graphicx} 
\usepackage{amsmath} %更多数学符号
\usepackage{wasysym}
\usepackage[unicode]{hyperref} %提供跳转链接

%%%% 正文开始 %%%%
\begin{document}
\makehead
\makebasis

对于习惯用 \LaTeX 写文档的同学们,一个写基金申请书的模版可能有参考作用。因此我做了这个2021年面上项目申请书正文部分模版。祝大家基金申请顺利!

2023-01-20: 根据2023年面上项目申请书正文的官方MsWord模板,对本模板的字号和少量蓝色文字做了更新。

2023-01-29\&02-01: 根据几位老师的建议,对section的缩进,“参考文献”四个字的大小、字体和居左等做了调整。官方模板中阿拉伯数字不加粗,因此也做了相应的调整。

\begin{figure}[!th]
\begin{center}
\includegraphics[width=1.8in]{fig-example.eps}
\caption{插图可以使用EPS、PNG、JPG等格式。}
\label{fig:example}
\end{center}
\end{figure}



\vskip 2mm
\subsubsection{1.1 声明}
{\bfseries \color{red} 注意!!!非国家自然科学基金委官方模版!!!}由个人根据官方MsWord模版制作。本模版的作者尽力使本模版和官方模版生成的PDF文件视觉效果大致一样,然而,并不保证本模版有用,也不对使用本模版造成的任何直接或间接后果负责。 不得将本模版用于商用或获取经济利益。本模版可以自由修改以满足用户自己的需要。但是如果要传播本模版,则只能传播未经修改的版本。使用本模版意味着同意上述声明。

如有问题,请发送邮件到 \href{mailto:ryanzz@foxmail.com}{ryanzz@foxmail.com}。中了基金的也欢迎反馈。$\smiley$

\subsubsection{1.2 使用说明}\label{sss:instruction}

\begin{enumerate}
\item 编译环境:推荐使用跨平台编译器texlive2017以后的版本,编译顺序为:xelatex+bibtex+xelatex(x2)。windows用户可以用命令行运行批处理文件getpdf.bat,linux用户可以运行runpdf。
\item 本模版力求简单,语句自身说明问题(self explanatory)。几乎只需要修改本tex文件即可满足排版需求,没有sty cls 等文件。用户掌握最基本的\LaTeX 语句即可操作,其余的可以用搜索引擎较容易地获得。
\item 参考文献样式:对作者个数作了限制以适合申请书,当作者个数小于等于5个时,予以全部保留,当作者个数大于5个时,只保留前3个,加et al。参考文献需要放在bib文件中。样式由ieeetrNSFC.bst控制。
\end{enumerate}

\subsubsection{1.3 图、公式和参考文献的引用示例}
尽管不大可能会用到像下面这样简单的公式:
\begin{equation}
\label{eq:ex}
\sqrt[15]{a}=\frac{1}{2},
\end{equation}
我们还是用Eq.(\ref{eq:ex})举个数学公式的例子。同时,我们也不大可能会用到一个长得很像\LaTeX 的图,但是还是引用一下图\ref{fig:example}。图\ref{fig:example}并没有告诉我们关于Jinkela\cite{John1997,Smith1900}的任何信息,也没有透露它的产地\cite{Piter1992}。尽管如此,最近的研究表明,Feizhou非常需要Jinkela\cite{John1997}。


%对作者个数作了限制以适合申请书
%当作者个数小于等于5个时,予以全部保留,当作者个数大于5个时,只保留3个,加et al

\bibliographystyle{ieeetrNSFC}
\bibliography{myexample}
\makecontent

\subsubsection{2.1 研究内容}
编制申请面上项目的\LaTeX 模版。

\subsubsection{2.2 研究目标}
帮助作者进行快速面上申请书的撰写、排版。

\subsubsection{2.3 拟解决的关键科学问题}
拟解决的{\bfseries 关键问题}包括:

\begin{itemize}
\item 中文的处理。
\item 参考文献\cite{John1997,Smith1900,Piter1992}的样式。
\item 官方word模版格式套用。
\item 符合\LaTeX 格式与内容分离的思想。
\end{itemize}

\makeproposal

\subsubsection{3.1 拟采取的研究方案}
详见1.2使用说明。

\subsubsection{3.2 可行性分析}
见pdf输出效果文件。

\makeinnovation

本模版修改自由度很高,可以按照用户自己的需求修改而不要求用户具有很多\LaTeX 技巧。

\makeplan

\subsubsection{5.1 年度研究计划}

\subsubsection{5.2 预期研究结果}
拟组织研讨会1次,将这个模版广而告之。但是目前还没有经费。

\makeresearchbase

申请人用\LaTeX 写过几篇文章,包括自己的博士论文。

\makerequirement

申请人课题组具有可以编译 \LaTeX 的计算机,可以成功编译此模版。

\makeongoing

无。

\makecompletion

不告诉你。

\makeotherprojects

无。

\makeapplicantconflict

无。

\makeongoingconflict

无。

\makeothers

无。
\end{document}


